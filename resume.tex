\section*{Résumé}
\addcontentsline{toc}{chapter}{\normalsize{Résumé}}
\small{\textbf{L}a chaleur émanant d’une bougie, l’air sortant d'un sèche-cheveux ou encore l’onde de choc produite par un avion entraînent des fluctuations de la densité optique. Celles-ci sont toutefois invisibles à l’œil nu, il faut donc concevoir des dispositifs d’imagerie afin de pouvoir les visualiser. Ce projet a porté sur l’étude d’un système d’imagerie Schlieren, dont le principe est similaire au filtrage du son : il s’agit de couper une partie des rayons déviés par un changement d’indice de réfraction afin d’agir sur la luminosité de l’image en sortie. L’équipement consiste en un miroir sphérique, dont le but est de concentrer la lumière d’une source ponctuelle, et d’une lame de rasoir en guise de filtre. L'effet de la source de chaleur est ensuite observé à l'aide d'un appareil photo. Le système conçu a donné des résultats satisfaisants : le contraste pourrait être amélioré, mais l'effet Schlieren est bien visible. L’objectif final de ce projet est de concevoir une onde de choc et de la visualiser à l'aide du dispositif optique.}
\\
\\
\small{\textbf{Mots-clés :} effet Schlieren, onde de choc, densité optique, indice de réfraction, filtre} 
