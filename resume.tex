\section*{Résumé}
\addcontentsline{toc}{chapter}{\normalsize{Résumé}}
\textbf{L}a chaleur émanant d’une bougie, l’air traversant un sèche-cheveux ou encore l’onde de choc produite par un avion entraînent des fluctuations de la densité optique. Celles-ci sont toutefois invisibles à l’œil nu, il faut donc concevoir des dispositifs d’imagerie afin de pouvoir les visualiser. Ce projet a porté sur l’étude d’un système d’imagerie Schlieren, c’est-à-dire un système basé sur le filtrage des rayons lumineux. Le principe est similaire au filtrage du son : il s’agit de couper une partie des rayons déviés par un changement d’indice de réfraction afin d’agir sur la luminosité de l’image en sortie. L’équipement consiste en un miroir sphérique, dont le but est de concentrer la lumière provenant d’une source ponctuelle, et d’une lame de rasoir en guise de filtre. La source de chaleur est placée devant le miroir et le résultat est observé à l’aide d’un appareil photo. Le système conçu a donné des résultats satisfaisants : même si le contraste de l’image obtenue pourrait être amélioré, l’effet Schlieren est bien visible sur l’appareil photo. L’objectif final de ce projet est de pouvoir appliquer ce dispositif à une onde de choc réalisée à partir de tubes PVC et d’une pompe à air.
