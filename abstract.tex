\section*{Abstract}
\addcontentsline{toc}{chapter}{\normalsize{Abstract}}
\textbf{H}eat emanating from a candle, air coming through a hairdryer or a shock wave produced by a plane create
fluctuations in optical density. However, they aren't visible to the naked eye; a specific system is needed
in order to observe and analyse these phenomena. Schlieren imaging systems are based on light filtering:
similarly to sound filtering, the purpose is to cut off part of the incoming light to create darker spots
where it has been deflected by a change in the refractive index of the air. The device that was set in place
consists of a spherical mirror that focuses the light coming from a point source and a razor blade that acts
as a filter. Once the components are all in place, the interfering object is set in front of the mirror and the
result is captured on camera. Experiments with matches gave pretty convincing results: although the
contrast and focus still need to be improved, the heat coming out was clearly visible on screen. The final
aim of this project is to generate a shockwave through a series of tubes directing air pressure and to
observe it with Schlieren photography.
\\
\\
\small{\textbf{Keywords :} Schlieren effect, shock wave, optical density, refractive index, filter}
