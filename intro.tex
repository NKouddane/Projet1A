\section*{Introduction}
\addcontentsline{toc}{chapter}{\normalsize{Introduction}}
\textbf{L}a visualisation des ondes de choc générées par les avions permet d’étudier leur mouvement et contribue aux recherches dans le domaine de l’aéronautique et au développement de nouveaux engins. L’observation du mouvement de l’air autour des appareils supersoniques peut être réalisée à l’aide d’un dispositif simple et efficace : le dispositif d’imagerie Schlieren. Celui-ci s’appuie sur les principes de base de transferts thermiques et d’optique géométrique,  principes utiles à tout étudiant ingénieur spécialisé en physique à Phelma. Ce projet constitue donc un moyen de mise en œuvre de connaissances théoriques pour la réalisation d’un livrable concret.
\\
\\
Cependant, l’enjeu du projet réside dans les différentes contraintes qui s’imposent à l’étudiant, notamment les contraintes budgétaires. Etant donné que le projet 1A est effectué en groupe, il nécessite un bon travail de planification, de coordination et de répartition des tâches. La deuxième difficulté majeure consiste à identifier correctement les causes d’un éventuel mauvais fonctionnement du dispositif et à proposer des pistes d’amélioration.
\\
\\
L’objectif de ce rapport est donc de présenter les différents moyens déployés afin de mener à bien ce projet, ce que soit sur le niveau technique ou organisationnel. Il présentera également une analyse détaillée des résultats obtenus, en plus des difficultés rencontrées et les améliorations effectuées.
\\
\\
Ainsi, ce document est réparti en trois parties : la première porte sur les aspects techniques du dispositif d’imagerie Schlieren et de l’onde de choc. La deuxième est, quant à elle, consacrée à tous les aspects de la gestion du projet. La troisième présente les résultats obtenus, les problèmes rencontrés et les solutions auxquels il y a eu recours afin de les résoudre. Enfin, une conclusion en guise de récapitulatif sera donnée à la fin du rapport.
