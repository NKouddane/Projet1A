\section*{Introduction}
\addcontentsline{toc}{chapter}{\normalsize{Introduction}}
\textbf{L}a visualisation des ondes de choc générées par les avions permet d’étudier leur mouvement et contribue aux recherches dans le domaine de l’aéronautique et au développement de nouveaux engins. L’observation du mouvement de l’air autour des appareils supersoniques peut être réalisée à l’aide d’un dispositif simple et efficace : le dispositif d’imagerie Schlieren. Celui-ci s’appuie sur les principes de base de transferts thermiques et d’optique géométrique,  principes utiles à tout étudiant en filière ingénierie physique. Ce projet constitue donc un moyen de mise en œuvre de connaissances théoriques pour la réalisation d’un livrable concret.
\\
\\
La problématique majeure du projet est la combinaison de deux systèmes distincts qui permettrait de visualiser et de photographier une onde de choc. Il faut en effet réaliser cette dernière ainsi que le dispositif de photographie par effet Schlieren.
\\
\\
L’objectif de ce rapport est donc de présenter non seulement les différents moyens déployés afin de mener à bien ce projet, que ce soit sur le niveau technique ou organisationnel, mais aussi les résultats obtenus lors de ce travail.
\\
\\
Ainsi, ce document est réparti en trois parties : la première porte sur les différents aspects du dispositif d’imagerie Schlieren. La deuxième est, quant à elle, consacrée à l'étude et la réalisation du dispositif à onde de choc. La troisième présente la combinaison des deux parties énoncées précédemment, et donc l'enjeu global du projet. Enfin, une conclusion en guise de récapitulatif sera donnée à la fin du rapport.
