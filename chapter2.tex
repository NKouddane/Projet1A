\renewcommand{\chaptername}{\scshape Partie}
\chapter{\normalfont \scshape Dispositif à onde de choc}
\section{Théorie du tube à choc}
\section{Protocole et organisation}
\subsection{Cahier des charges}
Après avoir déterminé les objectifs relatifs au dispositif à onde de choc, le diagramme de GANTT suivant a été établi :
\begin{figure}[H]
	\includegraphics[scale = 0.4]{figures/gantt_choc.png}
	\caption{\small{\textit{Diagramme de GANTT prévu pour le dispositif à onde de choc}}}
	\label{fig:gantt_choc}
\end{figure}
Le travail sur le tube à onde de choc a été confié aux trois membres restants du groupe. Le tableau~\ref{tab:gestion_choc} résume les postes attribués à chacun des membres:
\begin{table}[H]
	\centering
	\begin{tabular}{|l l l|}
		\hline
		\small\textbf{Responsable onde de choc}&\small\textbf{Responsable budget}&\small\textbf{Responsable technique}\\
		\hline
		\small{Hovanes BOKSYAN}&\small{Nino VIVIAND}&\small{Aymeric FREREJEAN}\\
		\hline
	\end{tabular}
	\caption{\small\textit{Membres et tâches attribuées (tube à onde de choc)}}
	\label{tab:gestion_choc}
\end{table}
\subsection{Mise en place du dispositif}
\section{Observations et conclusion}
\subsection{Description des résultats}
Afin de choisir le matériau adéquat pour la membrane, une série de tests a été effectuée. En effet, le matériau utilisé sur le site web consulté~\ref{ref:zigunov} qui a inspiré la conception du dispositif était difficilement trouvable en France. Par conséquent, une étude des matériaux dont les caractéristiques se rapprochent d’une feuille PVC de 70 µm a été réalisée.\\
Le tableau~\ref{tab:choc_resultats} résume les résultats obtenus selon le matériau utilisé pour la membrane.\\
\begin{table}[ht]
	\centering
	\begin{tabular}{|l|l|l|l|l|l|l|}
		\hline
		&\vtop{\hbox{\strut \small{Feuille de}}\hbox{\strut \small{plastifieuse}}\hbox{\strut \small{(e = 75 µm)}}}&\vtop{\hbox{\strut \small{Feuille}}\hbox{\strut \small{trans-}}\hbox{\strut \small{parente}}\hbox{\strut \small{(feuille de}}\hbox{\strut \small{classeur)}}}&\vtop{\hbox{\strut \small{Feuille de}}\hbox{\strut \small{papier}}\hbox{\strut \small{impri-}}\hbox{\strut \small{mante}}}&\vtop{\hbox{\strut \small{Mouchoirs}}\hbox{\strut \small{en papier}}}&\vtop{\hbox{\strut \small{Rouleau}}\hbox{\strut \small{adhésif}}\hbox{\strut \small{emballage}}\hbox{\strut \small{(ultra}}\hbox{\strut \small{résistant)}}}&\vtop{\hbox{\strut \bfseries\small{Rouleau}}\hbox{\strut \bfseries\small{adhésif}}\hbox{\strut \bfseries\small{type}}\hbox{\strut \bfseries\small{gaffer}}}\\
		\hline
		\vtop{\hbox{\strut \small{Pression}}\hbox{\strut \small{avant}}\hbox{\strut \small{rupture}}}&\vtop{\hbox{\strut \small{Pas de}}\hbox{\strut \small{rupture}}}&\vtop{\hbox{\strut \small{Pas de}}\hbox{\strut \small{rupture}}}&\vtop{\hbox{\strut \small{Pas de}}\hbox{\strut \small{rupture}}}&\small{< 1 bar}&\vtop{\hbox{\strut \small{Pas de}}\hbox{\strut \small{rupture}}}&\vtop{\hbox{\strut \small{\bfseries$\simeq$ 2,5}}\hbox{\strut \small{\small\bfseries bars}}}\\
		\hline
		\vtop{\hbox{\strut \small{Comm-}}\hbox{\strut \small{entaire}}}&\vtop{\hbox{\strut \small{Tests}}\hbox{\strut \small{jusqu'à 4}}\hbox{\strut \small{bars.}}\hbox{\strut \small{Fuites d'air}}\hbox{\strut \small{empêchant}}\hbox{\strut \small{l'augmentation}}\hbox{\strut \small{de la pression}}\hbox{\strut \small{(la feuille}}\hbox{\strut \small{se tord}}\hbox{\strut \small{sous la}}\hbox{\strut \small{contrainte.}}\hbox{\strut \small{Déformation}}\hbox{\strut \small{plastique}}\hbox{\strut \small{à partir}}\hbox{\strut \small{de 4 bars.}}}&\vtop{\hbox{\strut \small{Tests}}\hbox{\strut \small{jusqu'à 3}}\hbox{\strut \small{bars.}}\hbox{\strut \small{Trop de}}\hbox{\strut \small{déform-}}\hbox{\strut \small{ation}}\hbox{\strut \small{avant}}\hbox{\strut \small{rupture.}}\hbox{\strut \small{Matériau}}\hbox{\strut \small{trop}}\hbox{\strut \small{élastique.}}}&\vtop{\hbox{\strut \small{Tests}}\hbox{\strut \small{jusqu'à 3}}\hbox{\strut \small{bars.}}\hbox{\strut \small{Trop de}}\hbox{\strut \small{fuites d'air}}\hbox{\strut \small{car la}}\hbox{\strut \small{feuille}}\hbox{\strut \small{se tord}}\hbox{\strut \small{sous la}}\hbox{\strut \small{pression.}}}&\vtop{\hbox{\strut \small{Tests}}\hbox{\strut \small{réalisés}}\hbox{\strut \small{avec 1 à 8}}\hbox{\strut \small{couches.}}\hbox{\strut \small{Rupture}}\hbox{\strut \small{à trop}}\hbox{\strut \small{basse}}\hbox{\strut \small{pression}}\hbox{\strut \small{pour créer}}\hbox{\strut \small{une onde}}\hbox{\strut \small{de choc.}}}&\vtop{\hbox{\strut \small{Tests}}\hbox{\strut \small{jusqu'à 5}}\hbox{\strut \small{bars.}}\hbox{\strut \small{Pas de}}\hbox{\strut \small{rupture}}\hbox{\strut \small{avec une}}\hbox{\strut \small{seule}}\hbox{\strut \small{couche.}}\hbox{\strut \small{Imposs-}}\hbox{\strut \small{ible}}\hbox{\strut \small{d'augm-}}\hbox{\strut \small{enter la}}\hbox{\strut \small{pression}}\hbox{\strut \small{car trop}}\hbox{\strut \small{risqué.}}\hbox{\strut \small{Matériau}}\hbox{\strut \small{trop}}\hbox{\strut \small{résistant.}}}&\bfseries\vtop{\hbox{\strut \small{Tests}}\hbox{\strut \small{réalisés}}\hbox{\strut \small{avec}}\hbox{\strut \small{une seule}}\hbox{\strut \small{couche.}}\hbox{\strut \small{Pas de}}\hbox{\strut \small{fuite}}\hbox{\strut \small{et une}}\hbox{\strut \small{défrom-}}\hbox{\strut \small{ation du}}\hbox{\strut \small{matériau}}\hbox{\strut \small{quasi-}}\hbox{\strut \small{ment pas }}\hbox{\strut \small{visible,}}\hbox{\strut \small{donc la}}\hbox{\strut \small{rupture}}\hbox{\strut \small{est nette.}}}\\
		\hline
	\end{tabular}
	\caption{\small\textit{Résultats obtenus en fonction du matériau utilisé pour la membrane}}
	\label{tab:choc_resultats}
\end{table}
Le matériau le plus concluant s’est avéré être le scotch type “gaffer”, il a donc été conservé pour réaliser un test filmé. Les images suivantes ont été prises après avoir placé le canon du générateur à onde de choc en face d’un récipient rempli d’eau : \\
\begin{figure}[ht]
	\centering
	\includegraphics[scale = 0.5]{figures/choc_filmee.png}
	\caption{\small{\textit{Images du test de l'onde de choc générée}}}
	\label{fig:choc_filmee}
\end{figure}
\\
On constate que, même à une pression de 2,5 bars, le système génère une résultat souhaité. En effet, trois éléments témoignent du passage de l’onde de choc : tout d’abord, on observe de fortes éclaboussures, ce qui montre bien la vitesse élevée de sortie de l’air. Les mouvements des gouttes d’eau  sont accompagnés par un bruit sourd et de la fumée sortant du tuyau, comme le montre la dernière image de la figure~\ref{fig:choc_filmee}.\\\\
En empilant plusieurs couches de scotch, la pression avant rupture peut être augmentée jusqu’à 4 ou 5 bars, ce qui permet d’obtenir une onde de choc plus rapide. Cependant, une vitesse plus importante signifie une photographie plus difficile à visualiser à travers le dispositif d’imagerie Schlieren. De plus, le bruit serait encore plus sourd et nécessiterait alors le port d’un casque anti-bruit. 
\subsection{Interprétation}
\subsection{Conclusion partielle}