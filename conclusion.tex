$$\vcenter{}$$
\section*{Conclusion}
\addcontentsline{toc}{chapter}{\normalsize{Conclusion}}
\textbf{C}e rapport a eu comme objectif de présenter les différentes méthodes utilisées pour visualiser le mouvement d'un fluide à travers la photographie Schlieren, tout en exploitant celle-ci afin de visualiser une onde de choc. Il a également permis de comparer entre les différentes méthodes, choisir celle qui donnait le meilleur résultat, tester le tube à choc et définir les limites de l'imagerie Schlieren pour une onde de choc.\\\\
Il a été constaté, par exemple, que le miroir sphérique donnait un meilleur contraste étant donné qu'il permet une meilleure convergence des rayons lumineux. Le tube a choc conçu donne bien le résultat escompté, toutefois, les limites des capacités de l'appareil photo font que l'onde de choc n'est pas observable.
\\\\
Bien que l'observation du front d'onde soit impossible à cause de l'immense fréquence d'image requise, le projet a pu être mené à bien avec rigueur et en exploitant à leur plein potentiel les outils mis à disposition. Il ouvre toutefois des perspectives d'améliorations telles que l'exploitation de filtres de couleurs afin d'améliorer le contraste, ou encore l'automatisation par un système électronique de la capture d'image afin d'outrepasser les contraintes de fréquences d'images, en capturant une unique image à l'instant où le front d'onde se situe au centre du miroir.