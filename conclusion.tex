$$\vcenter{}$$
\section*{Conclusion}
\addcontentsline{toc}{chapter}{\normalsize{Conclusion}}
\textbf{C}e rapport a eu comme objectif de présenter les différentes méthodes utilisées pour visualiser le mouvement d'un fluide à travers la photographie Schlieren, tout en exploitant celle-ci afin de visualiser une onde de choc. Il a également permis de comparer entre les différentes méthodes, choisir celle qui donnait le meilleur résultat, tester le tube à choc et définir les limites de l'imagerie Schlieren pour une onde de choc.\\\\
Il a été constaté, par exemple, que le miroir sphérique donnait un meilleur contraste vu qu'il permet une meilleure convergence des rayons lumineux. Le tube a choc conçu donne bien le résultat souhaité, toutefois, les limites des capacités de l'appareil photo font que l'onde de choc n'est pas observable.
\\\\
Les limites des résultats obtenus ont été en grande partie liées aux caractéristiques des matériels utilisés ainsi qu'aux contraintes de budget. Ceci dit, le sujet de ce rapport peut donner lieu à d'autres perspectives qui pourraient être abordées dans d'autres projets, comme l'effet Schlieren avec filtrage en couleur, ou l'utilisation de ce type d'imagerie pour visualiser d'autres types d'ondes.